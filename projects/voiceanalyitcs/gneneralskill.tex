Based on our work together, here's a skill for the call center QA prompt development workflow:

```yaml
---
name: call-center-qa-prompt-editor
description: Develop, iterate, and validate LLM grading prompts for call center QA evaluation. Use when working on СО_2024 evaluation system, grading prompts, call evaluations, C-code criteria (C04-C17), or comparing prompt versions. Supports Russian and English documentation.
allowed-tools: Read, Grep, Glob, Write, Edit
---

# Call Center QA Prompt Editor

## Purpose

Assist in developing and refining LLM-based call center quality assurance grading prompts. The system evaluates operator calls against the СО_2024 standard (27 criteria, codes 10.1–2.2).

## Key Concepts

### Evaluation System Structure

| Phase | Criteria | Data Required |
|-------|----------|---------------|
| Phase 1 | 17 criteria (15 full + 2 partial) | Transcript only |
| Phase 2 | +5 criteria | + Audio analysis |
| Phase 3 | +5 criteria | + Oktell/CRM data |

### Critical Criteria (Most Common Violations)

| Code | СО Code | Description | Threshold |
|------|---------|-------------|-----------|
| C04 | 9.1 | Search duration | >40 sec = violation |
| C05 | 9.3 | Thanks after search | ANY search requires thanks |
| C06 | 7.1 | Script compliance | Greeting + closing elements |
| C07 | 7.2 | Echo method | Repeat + "верно?" for contacts |
| C08 | 7.3 | Response timing | >5 sec greeting delay = violation |

### Partial Assessment Criteria

- **C06 (7.1)**: Script compliance — only greeting/closing visible without full script files
- **C17 (2.2)**: Rudeness — only text profanity detectable without audio

## Workflow

### 1. Before Editing a Prompt

```
ALWAYS:
1. Read the Russian source document (СО_2024_ВТМ_полная.md)
2. Identify which criteria are affected by the change
3. Check existing interpretation notes (Интерпретации_и_примечания.md)
4. Review questions pending for client (Вопросы_к_автору.md)
```

### 2. Version Control Convention

```
Prompt versions: V{major}.{minor}.{patch}
- Major: Breaking changes to grading logic
- Minor: New criteria or significant rule changes  
- Patch: Clarifications, examples, edge cases

Example: V1.4.1 = 4th major iteration, 1st patch
```

### 3. Testing Changes

```
For any prompt change:
1. Identify affected calls from golden dataset (20 calls)
2. Re-run evaluation on affected calls
3. Compare grades: original vs new
4. Document any grade changes with rationale
```

### 4. Common Pitfalls to Avoid

**DO NOT invent rules not in source:**
- ❌ "10-second threshold for C05" (source says ANY search)
- ❌ "Interactive dialogue exception" (no exception in source)
- ❌ "C07 only for phone numbers" (applies to ALL contact data)

**DO cross-validate against Russian source:**
- Every rule must have a citation from СО_2024_ВТМ_полная.md
- When in doubt, flag for client confirmation

## JSON Output Schema

### Call Evaluation Format

```json
{
  "metadata": {
    "call_id": "XX",
    "operator": "Name",
    "call_type": "ORDER|INQUIRY|RESERVATION|INCIDENT",
    "duration_sec": 000,
    "phase": "Phase 1 (Transcript)"
  },
  "final_grade": 10|9|8|7|6|5|4|3|2,
  "grade_reason": "Lowest code principle: min(all violations)",
  "violations": [
    {
      "code": "C0X",
      "criterion": "X.X Description",
      "grade": 9,
      "status": "VIOLATION",
      "timestamp": "MM:SS-MM:SS",
      "evidence": "Russian text with exact quotes",
      "confidence": 0.85
    }
  ],
  "partial_assessments": [
    {
      "code": "C06|C17",
      "status": "GENERIC_ONLY|TEXT_ONLY",
      "finding": "What was observed",
      "limitation": "Why partial"
    }
  ],
  "criteria_passed": ["C01", "C02", ...],
  "summary": {
    "strengths": ["..."],
    "improvements": ["..."],
    "coaching": "Grade X. Brief coaching note."
  }
}
```

### call_type Classification

| Type | When to Use | C07 Applies? |
|------|-------------|--------------|
| ORDER | Order number assigned | ✅ Yes |
| RESERVATION | Booking number assigned | ✅ Yes |
| INCIDENT | Incident number assigned | ✅ Yes |
| INQUIRY | Consultation only, no transaction | ❌ No |

**Edge cases:**
- "Offered but declined" → INQUIRY
- "I'll order online myself" → INQUIRY
- "Data collection started, then cancelled" → INQUIRY (no number assigned)

## C06 Script Compliance (Detailed)

### What We CAN Evaluate (Phase 1)

**Greeting elements (all required):**
1. Company name ("74 колеса" / "Семьдесят четыре колеса")
2. Operator name
3. Offer to help / ask how to address customer

**Closing elements (required):**
1. "Чем ещё могу помочь?" or equivalent
2. Thanks for call
3. Farewell

### What We CANNOT Evaluate (needs script file)

- Project-specific question sequences
- Required upsells/cross-sells
- Specific information delivery order
- Custom script variations per project

### C06 Evaluation Logic

```
IF greeting missing company OR operator name:
  → C06 VIOLATION (Grade 7)
  
IF closing missing farewell after customer says goodbye:
  → C06 VIOLATION (Grade 7)
  
IF customer hangs up mid-operator-speech:
  → NOT a violation (customer initiated)
  
IF all basic elements present but can't verify full script:
  → status: "GENERIC_ONLY", no grade impact
```

## C05 Thanks After Search (Strict Rule)

### What Triggers C05

ANY announced search requires thanks, regardless of duration:
- "Минуту, пожалуйста"
- "Сейчас посмотрю"
- "Давайте проверим"
- "Секунду"

### Valid Thanks Phrases

✅ "Спасибо за ожидание"
✅ "Благодарю за ожидание"  
✅ "Спасибо, что подождали"

### NOT Thanks (Just Transitions)

❌ "Вот"
❌ "Так"
❌ "Нашла"
❌ "Хорошо, смотрите"

## C07 Echo Method (Detailed)

### Required Steps

1. Customer states data
2. Operator REPEATS the data
3. Operator asks "верно?" / "правильно?" OR uses question intonation
4. Customer confirms

### Applies To (when recording)

- Full name (ФИО)
- Phone number
- Address
- Email
- Any contact/identification data

### Does NOT Apply

- INQUIRY calls where name used conversationally
- Reusing existing data from previous orders (with customer confirmation)

## File Locations (Project Convention)

```
/project/
├── prompts/
│   └── V1_X_X_CallCenter_Grading_Prompt_EN.md
├── evaluations/
│   └── call_XX_FINAL.json
├── source_docs/
│   ├── СО_2024_ВТМ_полная.md (Russian source - authoritative)
│   ├── СО_2024_ВТМ_фаза1.md (Phase 1 subset)
│   └── Интерпретации_и_примечания.md
├── reports/
│   ├── CONSOLIDATED_REPORT.md
│   └── TECHNICAL_TABLE.md
└── questions/
    └── Вопросы_к_автору.md
```

## Changelog Format

When making prompt changes, document:

```markdown
## V1.X.X → V1.X.Y Changelog

| ID | Section | Change | Lines | Rationale |
|----|---------|--------|-------|-----------|
| 1 | C05 | Removed 10-sec threshold | 116 | Not in Russian source |
| 2 | C07 | Added call_type check | 180-185 | Only applies to ORDER/RESERVATION |
```

## Quality Checks Before Finalizing

- [ ] All rules traceable to СО_2024_ВТМ_полная.md
- [ ] No invented thresholds or exceptions
- [ ] Edge cases documented with examples
- [ ] Partial criteria clearly marked
- [ ] JSON schema followed exactly
- [ ] Russian evidence text preserved (not translated)
- [ ] Confidence levels calibrated (0.85-0.95 typical)
```

This skill captures our entire workflow. You can save it as `~/.claude/skills/call-center-qa-prompt-editor/SKILL.md` or in your project's `.claude/skills/` directory.